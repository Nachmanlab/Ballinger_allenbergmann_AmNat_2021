\PassOptionsToPackage{unicode=true}{hyperref} % options for packages loaded elsewhere
\PassOptionsToPackage{hyphens}{url}
%
\documentclass[]{article}
\usepackage{lmodern}
\usepackage{amssymb,amsmath}
\usepackage{ifxetex,ifluatex}
\usepackage{fixltx2e} % provides \textsubscript
\ifnum 0\ifxetex 1\fi\ifluatex 1\fi=0 % if pdftex
  \usepackage[T1]{fontenc}
  \usepackage[utf8]{inputenc}
  \usepackage{textcomp} % provides euro and other symbols
\else % if luatex or xelatex
  \usepackage{unicode-math}
  \defaultfontfeatures{Ligatures=TeX,Scale=MatchLowercase}
\fi
% use upquote if available, for straight quotes in verbatim environments
\IfFileExists{upquote.sty}{\usepackage{upquote}}{}
% use microtype if available
\IfFileExists{microtype.sty}{%
\usepackage[]{microtype}
\UseMicrotypeSet[protrusion]{basicmath} % disable protrusion for tt fonts
}{}
\IfFileExists{parskip.sty}{%
\usepackage{parskip}
}{% else
\setlength{\parindent}{0pt}
\setlength{\parskip}{6pt plus 2pt minus 1pt}
}
\usepackage{hyperref}
\hypersetup{
            pdfborder={0 0 0},
            breaklinks=true}
\urlstyle{same}  % don't use monospace font for urls
\usepackage[margin=1.0in]{geometry}
\usepackage{graphicx,grffile}
\makeatletter
\def\maxwidth{\ifdim\Gin@nat@width>\linewidth\linewidth\else\Gin@nat@width\fi}
\def\maxheight{\ifdim\Gin@nat@height>\textheight\textheight\else\Gin@nat@height\fi}
\makeatother
% Scale images if necessary, so that they will not overflow the page
% margins by default, and it is still possible to overwrite the defaults
% using explicit options in \includegraphics[width, height, ...]{}
\setkeys{Gin}{width=\maxwidth,height=\maxheight,keepaspectratio}
\setlength{\emergencystretch}{3em}  % prevent overfull lines
\providecommand{\tightlist}{%
  \setlength{\itemsep}{0pt}\setlength{\parskip}{0pt}}
\setcounter{secnumdepth}{0}
% Redefines (sub)paragraphs to behave more like sections
\ifx\paragraph\undefined\else
\let\oldparagraph\paragraph
\renewcommand{\paragraph}[1]{\oldparagraph{#1}\mbox{}}
\fi
\ifx\subparagraph\undefined\else
\let\oldsubparagraph\subparagraph
\renewcommand{\subparagraph}[1]{\oldsubparagraph{#1}\mbox{}}
\fi

% set default figure placement to htbp
\makeatletter
\def\fps@figure{htbp}
\makeatother

\usepackage{palatino}
\usepackage{setspace}
\doublespacing
\usepackage[left]{lineno}
\linenumbers

\author{}
\date{\vspace{-2.5em}}

\begin{document}

\hypertarget{environmental-and-genetic-contributions-to-ecogeographic-rules-in-a-wide-ranging-endotherm}{%
\section{Environmental and genetic contributions to ecogeographic rules
in a wide-ranging
endotherm}\label{environmental-and-genetic-contributions-to-ecogeographic-rules-in-a-wide-ranging-endotherm}}

\vspace{20mm}

Mallory A. Ballinger, Tifany Chu, and Michael W. Nachman\({^\dagger}\)

\vspace{40mm}

\({\dagger}\) To whom corresponsdence should be addressed:

\href{mailto:mnachman@berkeley.edu}{mnachman@berkeley.edu}

Department of Integrative Biology

Museum of Vertebrate Zoology

University of California, Berkeley

Berkeley, CA 94702-3160

\vspace{20mm}

\textbf{Running title:} Allen's Rule and Bergmann's Rule in house mice

\newpage

\hypertarget{abstract-200-words}{%
\subsection{Abstract (200 words)}\label{abstract-200-words}}

\newpage

\hypertarget{introduction}{%
\subsection{Introduction}\label{introduction}}

The mechanisms driving phenotypic diversity across and within species.

\hypertarget{clines-in-body-size-bergmanns-rule-and-extremity-length-allens-rule-have-historically-been-used-as-evidence-for-natural-selection-but-accurately-dissentangling-genetics-from-plasticity-is-not-possible-using-data-from-wild-caught-specimens.}{%
\subparagraph{\texorpdfstring{\textbf{Clines in body size (Bergmann's
Rule) and extremity length (Allen's Rule) have historically been used as
evidence for natural selection, but accurately dissentangling genetics
from plasticity is not possible using data from wild-caught
specimens.}}{Clines in body size (Bergmann's Rule) and extremity length (Allen's Rule) have historically been used as evidence for natural selection, but accurately dissentangling genetics from plasticity is not possible using data from wild-caught specimens.}}\label{clines-in-body-size-bergmanns-rule-and-extremity-length-allens-rule-have-historically-been-used-as-evidence-for-natural-selection-but-accurately-dissentangling-genetics-from-plasticity-is-not-possible-using-data-from-wild-caught-specimens.}}

Two of the most well described phenotypic clines are Bergmann's rule and
Allen's rule, with various support across both endotherms and
ectotherms. Strong clines in phenotypes (Huxley) have historically been
suggested to represent natural selection/local adaptation to
environments (Endler). Two very well described phenotypic clines are
Bergmann's rule and Allen's rule. These rules traditionally have been
used to describe phenotypes in endotherms, though there is evidence for
these rules in ectotherms. \textbf{Meta-analyses/syntheses focus on
``yes'' or ``no'' if Bergmann's rule is present, using various metrics
(e.g.~latitude, mean daily temp, etc); thus, the literautre is clouded
and we still do not know what mechanisms underly these rules.} Although
meta-analyses caputre a large number and variety of organisms, they also
capture a lot of noise. The positives about individual studies is that
researchers are likely to collect data in the same, careful way across
gradients.

\hypertarget{only-a-handful-of-studies-have-dissected-the-genetic-and-phenotypically-plastic-nature-of-bergmanns-rule-and-allens-rule-showing-plasticity-plays-a-large-role-underlying-these-classic-phenotypic-clines.-however-these-studies-only-investigated-clines-across-a-narrow-geographic-range.-reference-table-1}{%
\subparagraph{\texorpdfstring{\textbf{Only a handful of studies have
dissected the genetic and phenotypically plastic nature of Bergmann's
rule and Allen's rule, showing plasticity plays a large role underlying
these classic phenotypic clines. However, these studies only
investigated clines across a narrow geographic range.} (reference table
1)}{Only a handful of studies have dissected the genetic and phenotypically plastic nature of Bergmann's rule and Allen's rule, showing plasticity plays a large role underlying these classic phenotypic clines. However, these studies only investigated clines across a narrow geographic range. (reference table 1)}}\label{only-a-handful-of-studies-have-dissected-the-genetic-and-phenotypically-plastic-nature-of-bergmanns-rule-and-allens-rule-showing-plasticity-plays-a-large-role-underlying-these-classic-phenotypic-clines.-however-these-studies-only-investigated-clines-across-a-narrow-geographic-range.-reference-table-1}}

\hypertarget{house-mice-inhabit-a-very-broad-range-of-latitudes-across-the-americas-and-previous-research-has-shown-evidence-for-bergmanns-rule.-moreover-there-has-been-much-research-done-on-the-inheritent-plastic-nature-of-extremity-length-like-tails-and-ears-but-these-studies-only-investigated-single-populations-that-show-no-variation-in-starting-phenotypes.-overall-little-has-been-done-regarding-genetics-andor-plasticity-of-bergmannr-rule-and-allens-rule.}{%
\subparagraph{\texorpdfstring{\textbf{House mice inhabit a very broad
range of latitudes across the Americas, and previous research has shown
evidence for Bergmann's rule. Moreover, there has been much research
done on the inheritent plastic nature of extremity length (like tails
and ears), but these studies only investigated single populations that
show no variation in starting phenotypes. Overall, little has been done
regarding genetics and/or plasticity of Bergmann'r rule and Allen's
rule.}}{House mice inhabit a very broad range of latitudes across the Americas, and previous research has shown evidence for Bergmann's rule. Moreover, there has been much research done on the inheritent plastic nature of extremity length (like tails and ears), but these studies only investigated single populations that show no variation in starting phenotypes. Overall, little has been done regarding genetics and/or plasticity of Bergmann'r rule and Allen's rule.}}\label{house-mice-inhabit-a-very-broad-range-of-latitudes-across-the-americas-and-previous-research-has-shown-evidence-for-bergmanns-rule.-moreover-there-has-been-much-research-done-on-the-inheritent-plastic-nature-of-extremity-length-like-tails-and-ears-but-these-studies-only-investigated-single-populations-that-show-no-variation-in-starting-phenotypes.-overall-little-has-been-done-regarding-genetics-andor-plasticity-of-bergmannr-rule-and-allens-rule.}}

Previous evidence for Bergmann's rule in American house mice
(e.g.~Phifer-Rixey et al, 2018; Suzuki et al, 2020) was likely found
because (1) the overall small sample sizes that were collected, and (2)
the same research group collected the data in roughly the same manner.
House mice are great since they have very broad geographic/latitudinal
distribution across the Americas.

\hypertarget{in-order-to-dissect-the-genetic-and-plastic-underpinnings-of-these-ecogeographic-rules-first-have-variation-in-starting-populations-where-populations-are-different-and-presumably-locally-adapted-to-their-environments-and-perform-common-garden-experiments-to-dissentangle-genetics-from-plasticity.-here-we-use-house-mice-collected-from-ends-of-a-latitiudinal-cline-and-ask-about-the-genetic-and-plastic-underpinnings-to-these-broad-eco-geographic-rules.}{%
\subparagraph{\texorpdfstring{\textbf{In order to dissect the genetic
and plastic underpinnings of these ecogeographic rules, first have
variation in starting populations where populations are different (and
presumably locally adapted to their environments), and perform common
garden experiments to dissentangle genetics from plasticity. Here, we
use house mice collected from ends of a latitiudinal cline and ask about
the genetic and plastic underpinnings to these broad eco-geographic
rules.}}{In order to dissect the genetic and plastic underpinnings of these ecogeographic rules, first have variation in starting populations where populations are different (and presumably locally adapted to their environments), and perform common garden experiments to dissentangle genetics from plasticity. Here, we use house mice collected from ends of a latitiudinal cline and ask about the genetic and plastic underpinnings to these broad eco-geographic rules.}}\label{in-order-to-dissect-the-genetic-and-plastic-underpinnings-of-these-ecogeographic-rules-first-have-variation-in-starting-populations-where-populations-are-different-and-presumably-locally-adapted-to-their-environments-and-perform-common-garden-experiments-to-dissentangle-genetics-from-plasticity.-here-we-use-house-mice-collected-from-ends-of-a-latitiudinal-cline-and-ask-about-the-genetic-and-plastic-underpinnings-to-these-broad-eco-geographic-rules.}}

\newpage

\hypertarget{materials-and-methods}{%
\subsection{Materials and Methods}\label{materials-and-methods}}

\hypertarget{metadata-corresponding-to-bergmanns-rule-and-allens-rule-in-wild-caught-american-house-mice}{%
\subparagraph{Metadata corresponding to Bergmann's rule and Allen's rule
in wild-caught American house
mice}\label{metadata-corresponding-to-bergmanns-rule-and-allens-rule-in-wild-caught-american-house-mice}}

\newpage

\hypertarget{results}{%
\subsection{Results}\label{results}}

\hypertarget{weak-evidence-for-bergmanns-rule-and-allens-rule-in-wild-caught-american-house-mice.}{%
\subparagraph{\texorpdfstring{\textbf{1. Weak evidence for Bergmann's
rule and Allen's rule in wild-caught American house
mice.}}{1. Weak evidence for Bergmann's rule and Allen's rule in wild-caught American house mice.}}\label{weak-evidence-for-bergmanns-rule-and-allens-rule-in-wild-caught-american-house-mice.}}

{[}Weak patterns of Bergmann's rule (Figure 1A) and Allen's rule (Figure
1B; Figure S1) in North American and South American house mice.{]}{[}{]}
These data illustrate the inherent noisy nature of Bergmann's rule using
museum-collected data (e.g.~unknown specimen ages, variation in
collectors and measurements taken, temporal varation of metadata, etc.).
It is difficult to confidently say anything about the ``presence'' (or
``absence'') of Bergmann's rule/Allen's rule using museum metadata, let
alone the underlying mechanisms ``controlling/forming/driving/of'' these
clines (e.g.~temperature, precipitation, etc.). In order to reveal
morphological differences and dissentangle genetics and plasticity, a
common garden experiment needs to be performed, using mice from the ends
of the American latitudinal transect.

\hypertarget{bergmanns-rule-and-allens-rule-have-a-genetic-basis-in-american-house-mice.}{%
\subparagraph{\texorpdfstring{\textbf{2. Bergmann's rule and Allen's
rule have a genetic basis in American house
mice.}}{2. Bergmann's rule and Allen's rule have a genetic basis in American house mice.}}\label{bergmanns-rule-and-allens-rule-have-a-genetic-basis-in-american-house-mice.}}

{[}Population differences in body weight(Bergmann's Rule) (Figure 2A)
and tail length(Allen's Rule) (Figure 2B) persist in a common
environment across multiple generations, indicating a genetic
basis.{]}{[}{]} House mice collected from ends of latitudinal range are
significanlty different in body size and extremity length/tail length.
These differences persist across generations in a common environment,
thus indicating a genetic basis underlying these traits. Across
generations, there is a slight decrease in tail length in both
populations (with Brazil continuously having longer tails than New
York), perhaps suggesting an inherent plastic nature to this trait.
However, these data are also noisy (though not as noisy as VertNet), so
it is difficult to confidently ascertain the role of phenotypic
plasticity in this observation. In order to precisley measure the
impacts of plasticity on these ecogeographic traits, we must do a
\emph{controlled}, common garden experiemnt.

\hypertarget{developmental-plasticity-plays-a-signficant-role-in-forming-allens-rule-in-american-house-mice.}{%
\subparagraph{\texorpdfstring{\textbf{3. Developmental plasticity plays
a signficant role in ``forming'' Allen's rule in American house
mice.}}{3. Developmental plasticity plays a signficant role in ``forming'' Allen's rule in American house mice.}}\label{developmental-plasticity-plays-a-signficant-role-in-forming-allens-rule-in-american-house-mice.}}

{[}Unlike body weight, differences in tail length occur later in
development (Figure 3B) and are highly influenced by temperature, with
tails growing shorter in cold environments.{]}{[}{]} Evolved differences
in body size are evident at weaning and persist throughout development,
with New York mice being larger than Brazil mice across both sexes
(Figure 3A). Cold temperature has very little influence on the
development of body size. In contrast, tails start at \emph{roughly}
similar lengths at weaning and diverge into population-levels
differences later in development (Figure 3B). This variation in
developmental plasticity gives rise to populaiton-level differences in
tail length, with Brazil having longer tails than New York. Moreover,
unlike body size, tail length is highly plastic in response to cold
temperature, with mice in cold envrionments growing shorter tails,
regardless of population. The degree of this plasticity is more
pronouned in Brazil mice, with cold-reared Brazil mice growing much
shorter tails than warm-reared Brazil mice. In fact, the tail lenght of
cold-reared Brazil mice ``reaches'' the length of the evolved tail
length of New York mice.

\hypertarget{the-genetic-and-plastic-bases-of-bergmanns-rule-and-allens-rule-in-american-house-mice.}{%
\subparagraph{\texorpdfstring{\textbf{4.The genetic and plastic bases of
Bergmann's rule and Allen's rule in American house
mice.}}{4.The genetic and plastic bases of Bergmann's rule and Allen's rule in American house mice.}}\label{the-genetic-and-plastic-bases-of-bergmanns-rule-and-allens-rule-in-american-house-mice.}}

{[}Extremity length (tail length) is more plastic in response to cold
temperatures than is body size in American house mice.{]}{[}{]}
Differences in body size are mostly genetic, as body size shows little
plasticity across both populations and sexes (Figure 4A). This inherent
lack-of-plasticity in body size is not merely a result of increased
adiposity or differences in adiposity but instead is also seen at the
skeletal level (Fig Sx (correlation plots of skeletal trait vs body
length)). Tail length shows both a genetic basis and a plastic basis
(Figure 4B) in response to cold temperatures. Specifically, Brazil mice
have longer tails than New York mice, but when placed in a cold
environment, the tail length of Brazil mice is roughly the same tail
length of New York mice (Figure 4B). This plastic response seems to be
an exmaple of adaptive phentoypic plasticity. New York mice show very
little plasticity in both body size and tail length, presumably because
these traits are adatpive in a cold, temperate environment and/or tail
length in New York mice is canalized for the given temperature.

To explore the different genetic and plastic bases of tail length in New
York mice and Brazil mice, the number and length of caudal vertebrae
were measured.

\hypertarget{evolved-and-plastic-differences-in-the-number-and-length-of-caudal-vertebrae-underlie-genetic-and-plastic-bases-of-tail-length-allens-rule-in-american-house-mice.}{%
\subparagraph{\texorpdfstring{\textbf{5. Evolved and plastic differences
in the number and length of caudal vertebrae underlie genetic and
plastic bases of tail length (Allen's rule) in American house
mice.}}{5. Evolved and plastic differences in the number and length of caudal vertebrae underlie genetic and plastic bases of tail length (Allen's rule) in American house mice.}}\label{evolved-and-plastic-differences-in-the-number-and-length-of-caudal-vertebrae-underlie-genetic-and-plastic-bases-of-tail-length-allens-rule-in-american-house-mice.}}

{[}Brazil mice have more caudal vertebrae compared to New York mice
(Figure 5A), and there is population-level differences in plasticity of
these traits (Figure 5B).{]}{[}{]} ----- Need to anlayze this data
before elaborating on any further -----

\newpage

\hypertarget{discussion}{%
\subsection{Discussion}\label{discussion}}

Genetics of tail length differences (specifically vertebrae differences)
brought upon by selection for climbing ability? (More smaller vertebrae
is more flexible for climbing than just larger, fewer vertebrae) Future
research direction or question would be to assess if biomechanical
constraint also limits how much inherent plasticity can act on those
traits

\newpage

\hypertarget{acknowledgements}{%
\subsection{Acknowledgements}\label{acknowledgements}}

\newpage

\hypertarget{figures-tables}{%
\subsection{Figures \& Tables}\label{figures-tables}}

\textbf{Table 1. Studies dissecting the genetic and plastic bases of
ecogeographic rules in endotherms.}

\newpage

\textbf{Figure 1. Weak evidence for Bergmann's rule and Allen's rule in
wild-caught house mice across North and South America.}

\newpage

\textbf{Figure 2. Genetic basis of Bergmann's rule and Allen's rule in
American house mice.} N0 vs N1 vs N2 vs N3-N5??

!!!! ALLEN'S RULE AND LIMB-LENGTH !!!!

\newpage

\hypertarget{references}{%
\subsection{References}\label{references}}

\end{document}
